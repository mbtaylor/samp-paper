%% A&C uses 5p
\documentclass[5p]{elsarticle}

\usepackage{hyperref}
\usepackage{graphicx}

\journal{Astronomy \& Computing}

%% A&C uses model2-names (Harvard)
\bibliographystyle{model2-names}\biboptions{authoryear}

%%%%%%%%%%%%%%%%%%%%%%%

\begin{document}

\begin{frontmatter}

\title{SAMP, the Simple Applications Messaging Protocol}

%% Group authors per affiliation:
\author[bristol]{M.~B.~Taylor}
\ead{m.b.taylor@bristol.ac.uk}
\author[cds]{T.~Boch}
\author[google]{J.Taylor}

\address[bristol]{H.~H.~Wills Physics Laboratory, University of Bristol, UK}
\address[cds]{CDS, Observatoire Astronomique de Strasbourg, France}
\address[google]{Google, USA}

\begin{abstract}
SAMP, the Simple Applications Messaging Protocol, is a
hub-based communication standard for the exchange of data and control
between participating client applications.
It has been developed within the context of the Virtual Observatory
with the aim of enabling specialised data analysis tools to cooperate
as a loosely integrated suite, and is now in use by many and varied
desktop and web-based applications dealing with astronomical data.
This paper reviews the requirements and design principles that led to
SAMP's specification, provides a high-level description of the protocol,
and discusses some of its common and possible future usage patterns.
\end{abstract}

% is there a controlled/recommended vocabulary?
\begin{keyword}
SAMP
interoperability
publish-subscribe
\end{keyword}

\end{frontmatter}

\section{Introduction}

% try to avoid re-writing adass_xxi intro

Astronomical research requires complex and flexible manipulation
and processing of various different types of data.
Images, spectra, catalogues, time series, coverage maps and other data
types need their own special handling,
typically provided by specialist tools.
% and in many cases the required capabilities are provided by tools that
% specialise in a particular data type.
Data sets of different types meanwhile are usually related
in various ways arising from their physical origin,
for instance
catalogues are often derived from images and best understood in
conjunction with them, and
spectra and time series usually originate from specific sky positions
or regions which may be represented on images and described by
catalogue entries.
To extract scientific meaning from the data it is usually necessary
to exploit these linkages between data items as well as the
internal structure of each.

The working astronomer therefore uses a selection of different tools,
each specialising in a particular type of data or manipulation,
for different data sets and different tasks,
and has to integrate these together in a way that takes account of
the relationships of the data items under consideration.
For batch or pipeline-type processing this is usually, in terms of
data flow, fairly straightforward: the output of one step can be
fed to the input of the next as a file, stream of bytes, or some
kind of parameter list, often under the control of a script of some kind.

During the exploratory or interactive phase of data analysis however,
this traditional model of tool integration is less satisfactory.
Within a given GUI analysis
tool it is usual to interact with the data using
mouse and keyboard gestures to perform actions like selection or
navigation with instant visual feedback, in many cases with some
kind of internal linkage between different data views.
But communicating such actions or their results between different
tools tends to be much more cumbersome.
There is usually a way to reflect a result generated by one tool
in the state of another, for instance by reading sky coordinates
reported by one tool and typing or pasting them into another,
or saving an intermediate result from one tool to temporary storage
and reloading it into another, but it can be fiddly and tedious,
especially if similar actions are required repeatedly.
This lack of convenience is more than just an annoyance, it can
interrupt the flow of the data exploration, reduce the parameter
space able to be investigated, and effectively stifle discovery
of relationships present in the data.

% should there be a concrete example?
% is this stating the obvious?

From this point of view, a single monolithic astronomical data analysis
user application providing the best available facilities for
interactive presentation, manipulation and analysis of all kinds of
astronomical data and their interrelationships seems an attractive prospect.
In reality of course, no such one-stop analysis tool exists.
The obvious practical difficulties aside, it is not even clear
that deviating so far from the Unix philosophy of
``Make each program do one thing well'' \citep{mcilroy1978}
would be desirable.

These considerations have driven the development of a framework
for communication between independently-developed software items,
written in different languages and running in different processes.
Such applications can thus be made to appear to the user
as a loosely integrated suite of cooperating tools,
providing facilities such as data exchange, linked views and
remote control between each other.
Although communication between interactive desktop tools was the
original stimulus for what is now SAMP, the framework is flexible
enough to support other usage patterns as well.

The rest of this paper describes the history, design and use of SAMP.
Section
\ref{sec:history} outlines its evolution from the earlier PLASTIC protocol,
section \ref{sec:design} discusses the architectural principles and thinking
behind SAMP's design, and
section \ref{sec:protocol} provides a high-level view of the protocol itself.
Section \ref{sec:usage} gives some examples of its use in practice, and
section \ref{sec:conclusion} concludes by reviewing the current status
and possible future directions for the protocol.

\section{History} \label{sec:history}
 
SC4DEVO ... VOTech ... VODesktop ... AstroGrid Runtime ...
PLASTIC ... pub/sub ... XML-RPC/Java/RMILite
\citep{plastic_note}


\section{Design Principles} \label{sec:design}

\section{Protocol Description} \label{sec:protocol}

\section{Use in Practice} \label{sec:usage}

\section{Conclusions} \label{sec:conclusion}

\section*{References}

\bibliography{bibsamp}

\end{document}
