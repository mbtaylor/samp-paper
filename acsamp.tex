%% A&C uses 5p
\documentclass[5p]{elsarticle}

\usepackage{hyperref}
\usepackage{graphicx}

\journal{Astronomy \& Computing}

%% A&C uses model2-names (Harvard)
\bibliographystyle{model2-names}\biboptions{authoryear}

%%%%%%%%%%%%%%%%%%%%%%%

\begin{document}

\begin{frontmatter}

\title{SAMP, the Simple Applications Messaging Protocol}

%% Group authors per affiliation:
\author[bristol]{M.~B.~Taylor}
\ead{m.b.taylor@bristol.ac.uk}
\author[cds]{T.~Boch}
\author[google]{J.Taylor}

\address[bristol]{H.~H.~Wills Physics Laboratory, University of Bristol, UK}
\address[cds]{CDS, Observatoire Astronomique de Strasbourg, France}
\address[google]{Google}

\begin{abstract}
SAMP, the Simple Applications Messaging Protocol, is a
hub-based communication standard for the exchange of data and control
between participating client applications.
It has been developed within the context of the Virtual Observatory
with the aim of enabling specialised data analysis tools to cooperate
as a loosely integrated suite, and is now in use by many and varied
desktop and web-based applications dealing with astronomical data.
This paper reviews the requirements and design principles that led to
SAMP's specification, provides a high-level description of the protocol,
and discusses some of its common and possible future usage patterns.
\end{abstract}

\begin{keyword}
SAMP
interoperability
communication
messaging
publish-subscribe
\end{keyword}

\end{frontmatter}

\section{Introduction}

SAMP was a direct descendent of PLASTIC \citep{plastic_note}.

\section{Design Principles}

\section{Protocol Description}

\section{Use in Practice}

\section{Conclusions}

More research is required.

\section*{References}

\bibliography{bibsamp}

\end{document}
